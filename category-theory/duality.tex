\documentclass{atomic}
\usepackage{hyperref}

\title{Duality}
\newcommand{\breadcrumb}{Category Theory}

\begin{document}
    \pagestyle{plain} \currentdoc{note} \noindent
    We define the elementary language of \ex{Categories}{category theory} as the two sorted \ex{FirstOrderLogics}{first order logic} with
    \begin{itemize}
        \item objects and morphisms as the distinct sorts, together with 
        \item \ex{Relations}{relations} of an object being the source or target of a morphism, and
        \item a symbol for \ex{CompositeFunctions}{composition} morphisms.
    \end{itemize}
    Let $\sigma$ be a \ex{StatementsAndPropositions}{statement} in this language. The \textit{dual} of this statement $\sigma^\text{op}$ is the same as $\sigma$, except:
    \begin{itemize}
        \item The "source" and "target" \ex{Relations}{relations} are swapped, and
        \item The order of \ex{CompositeFunctions}{composition} is swapped (every $f \circ g$ becomes $g \circ f$).
    \end{itemize}

    \begin{definition}[Self-dual]
        A \ex{StatementsAndPropositions}{statement}/\ex{Sets}{set}/\ex{Classes}{class} is \textbf{self-dual} iff it is equal to its \textit{dual}.
    \end{definition}

    \begin{prop} \label{mirror-iff}
        $\sigma$ is true in $\mathcal{C}$ iff $\sigma^\text{op}$ is true in $\mathcal{C}^\text{op}$.
    \end{prop}

    \begin{theorem} \label{statements-in-self-dual-category-collections}
        If $\sigma$ holds for a \textbf{self-dual} \ex{Classes}{class} of \ex{Categories}{categories} $C$, so does $\sigma^\text{op}$.
    \end{theorem}
    \begin{proof}
        Let $C$ be a \ex{Classes}{class} of \ex{Categories}{categories} that is \textbf{self-dual}.
        Let $\sigma$ be a \ex{StatementsAndPropositions}{statement} in the elementary language of \ex{Categories}{category theory} that holds for all $\mathcal{C} \in C$.
        Since $C$ is \textbf{self-dual}, $\sigma$ holds for $\mathcal{C}^\text{op}$.
        Thus, by proposition \ref{mirror-iff} (and the fact that the dual of the dual is itself), $\sigma^\text{op}$ holds for $\mathcal{C}$. $\qed$
    \end{proof}


    
\end{document}
